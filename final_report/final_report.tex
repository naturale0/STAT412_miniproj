\documentclass[a4paper, 11pt, oneside]{article}
\usepackage[doublespacing]{setspace}
\usepackage[scale=0.75, twoside, bindingoffset=5mm]{geometry}
\usepackage[affil-it]{authblk}
\usepackage{enumerate}
\usepackage{hyperref}
\usepackage{amsmath}
\usepackage{graphicx}
\usepackage{csquotes}
\graphicspath{ {latex_figures/} }
\usepackage{booktabs,tabularx}
    \newcolumntype{L}{>{\raggedright\arraybackslash}X}
\usepackage{graphicx}
\usepackage{amssymb}


\title{Review of Multiple Comparison Correction methods\\
	\Large - From Bonferroni procedure to Efron's Local FDR - }
\author{2014140089 Park, Si Hyung}
\date{}

\begin{document}
\maketitle

\section{Introduction}
\flushleft
    Ever since works on multiple comparison problem (MCP) were started by Tukey (1949) and Scheff\'e (1953), who have both utilized $t-$statistic for pairwise sample mean comparisons, numerous procedures for controlling family-wise error rate (FWER) has been developed. Achieving higher testing power while controlling FWER for the same level has especially been a major concern in the field of MCP, rulling the field for almost 35 years before Benjamini and Hochberg suggested the concept of false discovery rate (FDR) in 1995. \par
    In this brief report, I would like to review two major concepts in MCP - FWER, FDR - and some of the selected procedures which controls FWER or FDR in certain level. Historical meanings and important proofs were described to better highlight modifications and developments of a statistical concept over time. Simple simulation of multiple testing was also conducted to compare different methods of MCP correction.

\subsection{Multiple comparison problem}
Multiple comparison problem (MCP), also known as multiple hypothesis tesing problem or multiplicity problem, is a problem which occurs when a family-level hypothesis testing is made based on a set of individual hypotheses. Due to the fact that a family-level type I error rate (family-wise error rate; FWER) is always greater than experiment-level type I error ($\because 1-(1-\alpha)^m > \alpha,$ if $m \geq 2$), an appropriate correction for the error control of family-level hypothesis testing is needed.

\subsection{Regression dependency of hypotheses}
For some family-level error controlling procedure, assumption on dependence structure is needed. Let $I_0$ be a subset of indices of hypotheses in a set of tests. If there exists no regression dependence between non-rejected test statistics and true-null test statistics, then the hypotheses in the set are said to be independent. If there is a positive regression dependence between non-rejected test statistics and true-null test statistics, then the hypotheses in the set are said to have positive regression dependency on each one from a subset $I_0$ ($PRDS$ on $I_0$). Benjamini and Yekutieli (2001) defined $PRDS$ on $I_0$ as follows: \enquote{For any increasing set $D$, and for each $i \in I_0$, $P(X \in D | X_i = x)$ is non-decreasing in $x$}. More intuitive explanation uses p-values. For each $i \in I_0$, If p-values increases, then probability of corresponding null hypothesis is true does not decreases.

\subsection{Notations}
\begin{table}[ht]
    \small
    \setlength{\tabcolsep}{3pt}
\centering
\begin{tabularx}{.7\hsize}{@{}l LLL@{}}
    \toprule
 & Declared as non-significant  & Declared as significant & Total \\
    \midrule
True $H_0$ 
    & $U$ 
        & $V$ 
            & $m_0$ \\
False $H_0$ 
    & $T$ 
        & $S$
            &  $m_1=m-m_0$ \\
Total & $m-R$ 
        & $R$ 
            & $m$ \\
    \bottomrule
\end{tabularx}
\caption{notation of the number of hypotheses in corresponding to each cell}
    \end{table}

Notations used to define family-level error concepts mathematically are as described on Table 1. This notation follows that of Benjamini and Hochberg (1995). Total $m$ hypotheses in a set is being tested in this situation. $m_0 \leq m$ hypotheses are true null. Note that $R$, the number of rejected hypotheses, is an observable random variable while $U$, $V$, $T$, $S$ are unobservable random variables.
\vspace{0.2in}


\section{Family-Wise Error Rate (FWER)}
Family-wise error rate is 
\subsection{Bonferroni procedure}
\subsection{Holm-Bonferroni procedure}
\subsection{Hochberg procedure}
\vspace{0.2in}


\section{False Discovery Rate (FDR)}
\subsection{Benjamini-Hochberg (B-H) procedure}
\subsection{Benjamini-Yekutieli (B-Y) procedure}
\subsection{Brief introduction to local FDR}
\vspace{0.2in}

\section{Discussions - Simulation and Comparisons}
\vspace{0.2in}

\section{References}
 Holm, S. (1979). "A Simple Sequentially Rejective Multiple Test Procedure". Scandinavian Journal of Statistics, 6(2), 65-70 \vspace{0.06in}\\
 Abdi, H. (2010). "Holm's Sequential Bonferroni Procedure". University of Texas at Dallas \vspace{0.06in}\\
 Aickin, M., Gensler, H. (1996). "Adjusting for multiple testing when reporting research results: the Bonferroni vs Holm methods". American Journal of Public Health, 86(5), 726-728 \vspace{0.06in}\\
 Benjamini, Y., Hochberg, Y. (1995). "Controlling the False Discovery Rate: A Practical and Powerful Approach to Multiple Testing". Journal of the Royal Statistical Society. Series B (Methodological), 57(1), 289-300 \vspace{0.06in}\\
 Benjamini, Y., Yekutieli, D. (2001). "The control of the false discovery rate in multiple testing under dependency". Ann. Statist. 29(4), 1165-1188 \vspace{0.06in}\\
 Robinson, D. (2015). "Understanding empirical Bayes estimation (using baseball statistics)". From \href{http://varianceexplained.org/r/empirical\_bayes\_baseball/}{Variance Explained}, retrieved at Dec. 2017 \vspace{0.06in}\\
 Efron, B., Tibshirani, R., Storey, J., Tusher, V. (2001). "Empirical Bayes Analysis of a Microarray Experiment". Stanford University Technical Report. 216 \vspace{0.06in}\\
 Efron, B. (2005). "Local False Discovery Rates". Stanford University Technical Report. 234 \vspace{0.06in}\\




\end{document}  